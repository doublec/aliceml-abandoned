\NeedsTeXFormat{LaTeX2e}
\documentclass[twoside]{article}
\usepackage{times,alltt,url,a4}

\setlength{\parskip}{1.5ex}
\setlength{\parindent}{0mm}

%%%%%%%%%%%%%%%%%%%%%%%%%%%%%%%%%%%%%%%%%%%%%%%%%%%%%%%%%%%%%%%%%%%%%%%%%%%%%%%%

\newcommand{\conarrow}{\hookrightarrow}

\newcommand{\x}[1]{\mathit{#1}}
\newcommand{\f}[1]{\mbox{#1}}
\renewcommand{\c}[1]{c_{\f{\scriptsize #1}}}

\newcommand{\lab}{\x{lab}}
\newcommand{\id}{\x{id}}
\newcommand{\longid}{\x{longid}}
\newcommand{\lit}{\x{lit}}
\renewcommand{\exp}{\x{exp}}
\newcommand{\exps}{\x{exps}}
\newcommand{\fld}{\x{fld}}
\newcommand{\flds}{\x{flds}}
\newcommand{\mat}{\x{mat}}
\newcommand{\mats}{\x{mats}}
\newcommand{\pat}{\x{pat}}
\newcommand{\pats}{\x{pats}}
\newcommand{\dec}{\x{dec}}
\newcommand{\decs}{\x{decs}}
\newcommand{\imp}{\x{imp}}
\newcommand{\imps}{\x{imps}}
\newcommand{\com}{\x{com}}

%%%%%%%%%%%%%%%%%%%%%%%%%%%%%%%%%%%%%%%%%%%%%%%%%%%%%%%%%%%%%%%%%%%%%%%%%%%%%%%%
\begin{document}
%%%%%%%%%%%%%%%%%%%%%%%%%%%%%%%%%%%%%%%%%%%%%%%%%%%%%%%%%%%%%%%%%%%%%%%%%%%%%%%%

\title{The Stockhausen Intermediate Language}
\author{Andreas Rossberg \\
Universit\"at des Saarlandes \\
\url{rossberg@ps.uni-sb.de}}
\date{\today}

\maketitle


%%%%%%%%%%%%%%%%%%%%%%%%%%%%%%%%%%%%%%%%%%%%%%%%%%%%%%%%%%%%%%%%%%%%%%%%%%%%%%%%
\section{Syntax}
\label{syntax}
%%%%%%%%%%%%%%%%%%%%%%%%%%%%%%%%%%%%%%%%%%%%%%%%%%%%%%%%%%%%%%%%%%%%%%%%%%%%%%%%

The syntax given here should be read as an abstract syntax. The meaning of most expressions should be straightforward. We briefly explain the non-obvious ones.

A long identifier expression $\longid.\lab$ is equivalent to the expression $(\f{lazy}\; \#\lab(\longid))$, ie. it represents lazy selection. The reason for having long identifiers separate is that they can appear in patterns.

The expression $(\f{new})$ generates a new constructor name for an extensible datatype. The long identifier in expressions of the form $@\longid(\exp)$ or patterns $@\longid(\pat)$ must denote such a generative constructor. On the other hand, $`\lab(\exp)$ and $`\lab(\pat)$ denote constructed values of closed (structural) datatypes. Strict expressions force evaluation of the argument. The dual strict pattern establishes the invariant that the value matched has been evaluated strictly and can thus not be a future. These constructs are particularly used to encode strictness of tags and constructor such that the backend can unbox them.

The expression $(\f{prim}\;s)$ is needed for binding primitive (builtin) values, which are named by a string. The $(\f{fail})$ expression marks an unused value. Forcing it always is an error. It is used during bootstrapping as a placeholder for runtime types.

The \f{seal} and \f{unseal} expressions are needed for typing purposes only and are described in section \ref{typing}.

\begin{center}
\begin{tabular}{lcll}
$\x{n}$	&$\in$&	\f{N}				& integer constants \\
$\x{w}$	&$\in$&	\f{W}				& word constants \\
$\x{c}$	&$\in$&	\f{C}				& character constants \\
$\x{s}$	&$\in$&	\f{S}				& string constants \\
$\x{p}$	&$\in$&	\f{P}				& float constants \\
$\id$	&$\in$&	\f{Id}				& identifiers \\
$\lab$	&$\in$&	$\f{Lab}=\f{Id}\uplus\f{N}_+$	& labels \\
\ \\

$\lit$	&$:=$&	$n$				& integer literal \\
	&&	$\x{w}$				& word \\
	&&	$\x{c}$				& character \\
	&&	$\x{s}$				& string \\
	&&	$\x{p}$				& float \\
\ \\

$\longid$	&$:=$&	$\id$			& short identifier \\
	&&	$\longid.\lab$			& long \\
\end{tabular}
\end{center}

\begin{center}
\begin{tabular}{lcll}

$\exps$	&$:=$&	$\exp_1,\cdots,\exp_n$		& expressions $(n\geq0)$ \\
$\exp$	&$:=$&	$\lit$				& literal expression \\
	&&	$\f{prim}\;\x{s}$		& primitive value \\
	&&	`$v$'				& immediate value \\
	&&	$\f{new}$			& constructor generation \\
	&&	$\longid$			& variable \\
	&&	$\#\lab(\exp)$			& product projection \\
	&&	$`\lab(\exp)$			& sum injection \\
	&&	$@\longid(\exp)$		& construction \\
	&&	$\f{ref}\;\exp$			& reference \\
	&&	$\f{roll}\;\exp$		& recursive typing \\
	&&	$\f{strict}\;\exp$		& strict evaluation \\
	&&	$[\exps]$			& vector \\
	&&	$(\exps)$			& tuple \\
	&&	$\{\flds^\exp\}$		& labelled product \\
	&&	$\f{fun}\;\mats$		& function \\
	&&	$\exp_1(\exp_2)$		& application \\
	&&	$\exp_1\;\f{and}\;\exp_2$	& conjunction \\
	&&	$\exp_1\;\f{or}\;\exp_2$	& disjunction \\
	&&	$\f{if}\;\exp_1\;\f{then}\;\exp_2\;\f{else}\;\exp_3$
						& conditional \\
	&&	$\exp_1;\exp_2$			& sequencing \\
	&&	$\f{case}\;\exp\;\f{of}\;\mats$ & case switch \\
	&&	$\f{raise}\;\exp$		& exception raising \\
	&&	$\f{try}\;\exp\;\f{handle}\;\mats$ & exception handling \\
	&&	$\f{fail}$			& lazy failure \\
	&&	$\f{lazy}\;\exp$		& lazy evaluation \\
	&&	$\f{spawn}\;\exp$		& concurrent evaluation \\
	&&	$\f{let}\;\decs\;\f{in}\;\exp$	& local declarations \\
	&&	$\f{seal}\;\exp$		& sealing \\
	&&	$\f{unseal}\;\exp$		& unsealing \\
\ \\

$\flds^x$&$:=$&	$\fld^x_1,\cdots,\fld^x_n$	& row $(n\geq0)$ \\
$\fld^x$ &$:=$& $\lab = x$			& field \\
\ \\

$\mats$	&$:=$&	$\mat_1\;|\cdots|\;\mat_n$	& matches $(n\geq1)$ \\
$\mat$	&$:=$&	$\pat\to\exp$			& match \\
\ \\

$\pats$	&$:=$&	$\pat_1,\cdots,\pat_n$		& patterns $(n\geq0)$ \\
$\pat$	&$:=$&	$\lit$				& literal pattern \\
	&&	$\f{\_}$			& joker \\
	&&	$\id$				& variable \\
	&&	$`\lab(\pat)$			& sum injection \\
	&&	$@\longid(\pat)$		& construction \\
	&&	$\f{ref}\;\pat$			& reference \\
	&&	$\f{roll}\;\pat$		& recursive typing \\
	&&	$\f{strict}\;\pat$		& non-future pattern \\
	&&	$[\pats]$			& vector \\
	&&	$(\pats)$			& tuple \\
	&&	$\{\flds^\pat,\_\}$		& labelled product \\
	&&	$\pat_1\;\f{as}\;\pat_2$	& conjunctive \\
	&&	$\pat_1\;|\;\pat_2$		& disjunctive \\
	&&	$\f{non}\;\pat$			& negative \\
	&&	$\pat\;\f{where}\;\exp$		& guarded \\
	&&	$\pat\;\f{with}\;\decs$		& local declarations \\
\ \\

$\decs$	&$:=$&	$\dec_1;\cdots;\dec_n$		& declarations $(n\geq0)$ \\
$\dec$	&$:=$&	$\pat = \exp$			& value declaration \\
	&&	$\f{rec}\;\decs$		& recursive declarations \\
\end{tabular}
\end{center}

\begin{center}
\begin{tabular}{lcll}

$\imps$	&$:=$&	$\imp_1;\cdots;\imp_n$		& imports $(n\geq0)$ \\
$\imp$	&$:=$&	$\id\;\f{from}\;s$		& import \\
\ \\

$\com$	&$:=$&	$\f{import}\;\imps\;\f{let}\;\decs\;\f{in}\;\flds^\id$
						& component

\end{tabular}
\end{center}

Note that $\fld^x$ and $\flds^x$ actually stand for a family of symbols with different instantiations of the symbol $x$ used in their productions.


%%%%%%%%%%%%%%%%%%%%%%%%%%%%%%%%%%%%%%%%%%%%%%%%%%%%%%%%%%%%%%%%%%%%%%%%%%%%%%%%
\section{Types}
\label{types}
%%%%%%%%%%%%%%%%%%%%%%%%%%%%%%%%%%%%%%%%%%%%%%%%%%%%%%%%%%%%%%%%%%%%%%%%%%%%%%%%

The following grammar defines the set $\f{T}$ of {\em types}. We omit existential types since we do not require them yet. Moreover, we omit higher kinds.

\begin{center}
\begin{tabular}{lcll}
$\alpha$&$\in$&	\f{V}				& variables \\
$l$	&$\in$&	\f{A}				& labels \\
$c$	&$\in$&	\f{C}				& type names \\
\ \\

$t$	&$:=$&	$\alpha$			& type variable \\
	&&	$c$				& type name \\
	&&	$t_1 \to t_2$			& arrow \\
	&&	$t_1 \conarrow t_2$		& constructor arrow \\
	&&	$\{r\}$				& product \\
	&&	$[\,r\,]$			& sum \\
	&&	$\mu \alpha . t$		& recursive \\
	&&	$\forall \alpha . t$		& universal quantification \\
%	&&	$\exists \alpha . t$		& existential quantification \\
%	&&	$\lambda \alpha:k . t$		& function \\
%	&&	$t_1(t_2)$			& application \\
\ \\

$r$	&$:=$&	$\epsilon$			& empty row \\
	&&	$l:t,r$				& field \\
%\ \\
%
%$k$	&$:=$&	$*$				& ground type \\
%	&&	$+$				& extensible type \\
%	&&	$k_1 \to k_2$			& constructor \\
\end{tabular}
\end{center}

We assume the following additional equivalence rules:
\begin{eqnarray*}
\mu \alpha.t &=& \mu \alpha.t[\mu \alpha.t/\alpha] \\
\forall \alpha_1.t &=& \forall \alpha_2.t[\alpha_2/\alpha_1] \\
\forall \alpha_1.\forall \alpha_2.t &=& \forall \alpha_2.\forall \alpha_1.t \\
l_1:t_1,l_2:t_2,r &=& l_2:t_2,l_1:t_1,r
\end{eqnarray*}
The first rule is known is {\em Shao's equation} and allows expressing mutually recursive types without requiring a polyadic fixpoint operator. It comes down to a slightly more powerful variation of iso-recursion.

We write $t_1 \succ t_2$ if $t_1$ matches $t_2$, i.e.\ if $t_2$ is an instantiation of the possibly quantified type $t_1$. Matching does not imply that $t_1$ is quantifier-free.

Type names $c$ carry an extensibility flag which is true only for extensible sum types.\footnote{In the implementation this is part of the kind system.} First-class constructors for an extensible sum type $t$ are assigned constructor arrow types $t' \conarrow t$ where $t'$ is the argument type and $t = c(t_1)\cdots(t_n)$ for some $n\geq0$.

An {\em environment} is a finite mapping $E \in \f{Id}\to\f{T}$.


%%%%%%%%%%%%%%%%%%%%%%%%%%%%%%%%%%%%%%%%%%%%%%%%%%%%%%%%%%%%%%%%%%%%%%%%%%%%%%%%
\section{Typing Rules}
\label{typing}
%%%%%%%%%%%%%%%%%%%%%%%%%%%%%%%%%%%%%%%%%%%%%%%%%%%%%%%%%%%%%%%%%%%%%%%%%%%%%%%%

Every expression $\exp$, pattern $\pat$, identifier $\id$, or long identifier $\longid$ is annotated with a type. In the rules' conclusions we denote that annotation by $:t$, respectively. The rules assume well-formedness of annotation types.

Note that identifiers themselves are annotated with their full polymorphic type, while identifier expressions and patterns are annotated with the corresponding instantiation type.

Note also that the rules enforce that there is no shadowing of bound identifiers. In fact we demand an even stronger invariant which is not captured by the rules, namely that all bound identifiers have to be disjoint. The only exception are disjunctive patterns where every arm must contain the same bound identifiers.

Any expression that does not contain \f{prim} or \f{seal} and \f{unseal} expressions is type-safe. Sealing and unsealing are sort of weak casts that only allow constructor names to be replaced by concrete types in one direction. They are used for translated module expressions (sealing, functor application, and packages). The rules for these expressions just require the existence of an appropriate realisation $\varphi$ (a mapping from constructor names to types). Unfortunately, this is undecidable in general since it requires higher-order unification. The current verifier uses an approximation that is correct for first-order cases and accepts any higher-order ones. This can be repaired by annotating the expressions with the actual substitution which is known at translation time.

The reason we use casts is that the presence of first-class signatures in Stockhausen requires a more expressive type system --- without first-class signatures modules could be translated to the intermediate language in a sound way, as shown by Russo. However, we would need polymorphic kinds and record kinds to express first-class signatures. For packages we need some sort of cast anyway.


\subsubsection*{Literals \hfill
\fbox{$\vdash \lit \Rightarrow t$}
}

\begin{equation}
\frac{}{
\vdash n \Rightarrow \c{int}
}
\end{equation}

\begin{equation}
\frac{}{
\vdash w \Rightarrow \c{word}
}
\end{equation}

\begin{equation}
\frac{}{
\vdash c \Rightarrow \c{char}
}
\end{equation}

\begin{equation}
\frac{}{
\vdash s \Rightarrow \c{string}
}
\end{equation}

\begin{equation}
\frac{}{
\vdash p \Rightarrow \c{float}
}
\end{equation}



\subsubsection*{Labels \hfill
\fbox{$\vdash \lab \Rightarrow l$}
}

\begin{equation}
\frac{
[\![\lab]\!] = l
}{
\vdash \lab \Rightarrow l
}
\end{equation}



\subsubsection*{Binding Identifiers \hfill
\fbox{$\vdash \id \Rightarrow t,E$}
}

\begin{equation}
\frac{}{
\vdash \id : t \Rightarrow t,\{\id \mapsto t\}
}
\end{equation}



\subsubsection*{Identifiers \hfill
\fbox{$E \vdash \id \Rightarrow t$}
}

\begin{equation}
\frac{
E(\id) = t
}{
E \vdash \id : t \Rightarrow t
}
\end{equation}



\subsubsection*{Long Identifiers \hfill
\fbox{$E \vdash \longid \Rightarrow t$}
}

\begin{equation}
\frac{
E \vdash \id \Rightarrow t
}{
E \vdash \id : t \Rightarrow t
}
\end{equation}

\begin{equation}
\frac{
E \vdash \lab \Rightarrow l
\qquad
E \vdash \longid \Rightarrow \{l:t,r\}
}{
E \vdash \longid.\lab : t \Rightarrow t
}
\end{equation}



\subsubsection*{Expression Sequences \hfill
\fbox{$E \vdash \exps \Rightarrow t_1,\cdots,t_n$}
}

\begin{equation}
\frac{
E \vdash \exp_i \Rightarrow t_i
}{
E \vdash \exp_1,\cdots,\exp_n \Rightarrow t_1,\cdots,t_n
}
\end{equation}



\subsubsection*{Expressions \hfill
\fbox{$E \vdash \exp \Rightarrow t$}
}

\begin{equation}
\frac{
\vdash \lit \Rightarrow t
}{
E \vdash \lit : t \Rightarrow t
}
\end{equation}

\begin{equation}
\frac{
%t : *
}{
E \vdash \f{prim}\;s : t \Rightarrow t
}
\end{equation}

\begin{equation}
\frac{
%t : *
}{
E \vdash \mbox{`$v$'} : t \Rightarrow t
}
\end{equation}

\begin{equation}
\frac{
t = t_0 \conarrow c(t_1)\cdots(t_n)
\qquad
\mbox{$c$ extensible}
}{
E \vdash \f{new} : t \Rightarrow t
}
\end{equation}

\begin{equation}
\frac{
E \vdash \longid \Rightarrow t'
\qquad
t' \succ t
}{
E \vdash \longid : t \Rightarrow t
}
\end{equation}

\begin{equation}
\frac{
\vdash \lab \Rightarrow l
\qquad
E \vdash \exp \Rightarrow \{l:t,r\}
}{
E \vdash \#\lab(\exp) : t \Rightarrow t
}
\end{equation}

\begin{equation}
\frac{
\vdash \lab \Rightarrow l
\qquad
E \vdash \exp \Rightarrow t'
\qquad
t = [l:t',r]
}{
E \vdash `\lab(\exp) : t \Rightarrow t
}
\end{equation}

\begin{equation}
\frac{
\begin{array}{@{}c@{}}
E \vdash \longid \Rightarrow t_1
\qquad
E \vdash \exp \Rightarrow t_2
\\
t_1 \succ t_2 \conarrow t
\qquad
t = c(t_1')\cdots(t_n')
\qquad
\mbox{$c$ extensible}
\end{array}
}{
E \vdash @\longid(\exp) : t \Rightarrow t
}
\end{equation}

\begin{equation}
\frac{
E \vdash \exp \Rightarrow t'
\qquad
t = \c{ref}(t')
}{
E \vdash \f{ref}\;\exp : t \Rightarrow t
}
\end{equation}

\begin{equation}
\frac{
E \vdash \exp \Rightarrow t'[t/\alpha]
\qquad
t = \mu \alpha.t'
}{
E \vdash \f{roll}\;\exp : t \Rightarrow t
}
\end{equation}

\begin{equation}
\frac{
E \vdash \exp \Rightarrow t'
\qquad
t' \notin V \cup C
\qquad
t = \c{strict}(t')
}{
E \vdash \f{strict}\;\exp : t \Rightarrow t
}
\end{equation}

\begin{equation}
\frac{
E \vdash \exps \Rightarrow t',\cdots,t'
\qquad
t = \c{vec}(t')
}{
E \vdash [\exps] : t \Rightarrow t
}
\end{equation}

\begin{equation}
\frac{
E \vdash \exps \Rightarrow t_1,\cdots,t_n
\qquad
t = \{1:t_1,\cdots,n:t_n,\epsilon\}
}{
E \vdash (\exps) : t \Rightarrow t
}
\end{equation}

\begin{equation}
\frac{
E \vdash \flds^\exp \Rightarrow r
\qquad
t = \{r\}
}{
E \vdash \{\flds^\exp\} : t \Rightarrow t
}
\end{equation}

\begin{equation}
\frac{
E \vdash \mats \Rightarrow t_1,t_2
\qquad
t = t_1 \to t_2
}{
E \vdash \f{fun}\;\mats : t \Rightarrow t
}
\end{equation}

\begin{equation}
\frac{
E \vdash \exp_1 \Rightarrow t' \to t
\qquad
E \vdash \exp_2 \Rightarrow t'
}{
E \vdash \exp_1(\exp_2) : t \Rightarrow t
}
\end{equation}

\begin{equation}
\frac{
E \vdash \exp_1 \Rightarrow \c{bool}
\qquad
E \vdash \exp_2 \Rightarrow \c{bool}
\qquad
t = \c{bool}
}{
E \vdash \exp_1\;\f{and}\;\exp_2 : t \Rightarrow t
}
\end{equation}

\begin{equation}
\frac{
E \vdash \exp_1 \Rightarrow \c{bool}
\qquad
E \vdash \exp_2 \Rightarrow \c{bool}
\qquad
t = \c{bool}
}{
E \vdash \exp_1\;\f{or}\;\exp_2 : t \Rightarrow t
}
\end{equation}

\begin{equation}
\frac{
E \vdash \exp_1 \Rightarrow \c{bool}
\qquad
E \vdash \exp_2 \Rightarrow t
\qquad
E \vdash \exp_3 \Rightarrow t
}{
E \vdash \f{if}\;\exp_1\;\f{then}\;\exp_2\;\f{else}\;\exp_3 : t \Rightarrow t
}
\end{equation}

\begin{equation}
\frac{
E \vdash \exp_1 \Rightarrow t'
\qquad
E \vdash \exp_2 \Rightarrow t
}{
E \vdash \exp_1;\exp_2 : t \Rightarrow t
}
\end{equation}

\begin{equation}
\frac{
E \vdash \exp \Rightarrow t'
\qquad
E \vdash \mats \Rightarrow t',t
}{
E \vdash \f{case}\;\exp\;\f{of}\;\mats : t \Rightarrow t
}
\end{equation}

\begin{equation}
\frac{
E \vdash \exp \Rightarrow \c{exn}
%\qquad
%t : *
}{
E \vdash \f{raise}\;\exp : t \Rightarrow t
}
\end{equation}

\begin{equation}
\frac{
E \vdash \exp \Rightarrow t
\qquad
E \vdash \mats \Rightarrow \c{exn},t
}{
E \vdash \f{try}\;\exp\;\f{handle}\;\mats : t \Rightarrow t
}
\end{equation}

\begin{equation}
\frac{
%t : *
}{
E \vdash \f{fail} : t \Rightarrow t
}
\end{equation}

\begin{equation}
\frac{
E \vdash \exp \Rightarrow t
}{
E \vdash \f{lazy}\;\exp : t \Rightarrow t
}
\end{equation}

\begin{equation}
\frac{
E \vdash \exp \Rightarrow t
}{
E \vdash \f{spawn}\;\exp : t \Rightarrow t
}
\end{equation}

\begin{equation}
\frac{
E \vdash \decs \Rightarrow E'
\qquad
E \uplus E' \vdash \exp \Rightarrow t
}{
E \vdash \f{let}\;\decs\;\f{in}\;\exp : t \Rightarrow t
}
\end{equation}

\begin{equation}
\frac{
E \vdash \exp \Rightarrow t'
\qquad
\exists \varphi . \varphi(t) = t'
}{
E \vdash \f{seal}\;\exp : t \Rightarrow t
}
\end{equation}

\begin{equation}
\frac{
E \vdash \exp \Rightarrow t'
\qquad
\exists \varphi . \varphi(t') = t
}{
E \vdash \f{unseal}\;\exp : t \Rightarrow t
}
\end{equation}



\subsubsection*{Field Sequences \hfill
\fbox{$E \vdash \flds^x \Rightarrow r\langle,E'\rangle$}
}

\begin{equation}
\frac{
\begin{array}{@{}c@{}}
E \vdash \fld^x_i \Rightarrow l_i,t_i\langle,E_i\rangle
\qquad
\mbox{$l_i$ disjoint}
\\
r = l_1:t_1,\cdots,l_n:t_n,\epsilon
\qquad
\langle E' = \biguplus E_i \rangle
\end{array}
}{
E \vdash \fld^x_1,\cdots,\fld^x_n \Rightarrow r\langle,E'\rangle
}
\end{equation}


\subsubsection*{Fields \hfill
\fbox{$E \vdash \fld^x \Rightarrow l,t\langle,E'\rangle$}
}

\begin{equation}
\frac{
\vdash \lab \Rightarrow l
\qquad
E \vdash x \Rightarrow t\langle,E'\rangle
}{
E \vdash \lab = x \Rightarrow l,t\langle,E'\rangle
}
\end{equation}



\subsubsection*{Match Sequences \hfill
\fbox{$E \vdash \mats \Rightarrow t_1,t_2$}
}

\begin{equation}
\frac{
E \vdash \mat_i \Rightarrow t_1,t_2
}{
E \vdash \mat_1\;|\cdots|\;\mat_n \Rightarrow t_1,t_2
}
\end{equation}



\subsubsection*{Matches \hfill
\fbox{$E \vdash \mat \Rightarrow t_1,t_2$}
}

\begin{equation}
\frac{
E \vdash \pat \Rightarrow t_1,E'
\qquad
E \uplus E' \vdash \exp \Rightarrow t_2
}{
E \vdash \pat\to\exp \Rightarrow t_1,t_2
}
\end{equation}



\subsubsection*{Pattern Sequences \hfill
\fbox{$E \vdash \pats \Rightarrow t_1,\cdots,t_n,E'$}
}

\begin{equation}
\frac{
E \vdash \pat_i \Rightarrow t_i,E_i
\qquad
E' = \biguplus E_i
}{
E \vdash \pat_1,\cdots,\pat_n \Rightarrow t_1,\cdots,t_n,E'
}
\end{equation}



\subsubsection*{Patterns \hfill
\fbox{$E \vdash \pat \Rightarrow t,E'$}
}

\begin{equation}
\frac{
\vdash \lit \Rightarrow t
}{
E \vdash \lit : t \Rightarrow t,\{\}
}
\end{equation}

\begin{equation}
\frac{
%t : *
}{
E \vdash \f{\_} : t \Rightarrow t,\{\}
}
\end{equation}

\begin{equation}
\frac{
\vdash \id \Rightarrow t',E'
\qquad
t' \succ t
}{
E \vdash \id : t \Rightarrow t,E'
}
\end{equation}

\begin{equation}
\frac{
\vdash \lab \Rightarrow l
\qquad
E \vdash \pat \Rightarrow t',E'
\qquad
t = \{l:t',r\}
}{
E \vdash `\lab(\pat) : t \Rightarrow t,E'
}
\end{equation}

\begin{equation}
\frac{
\begin{array}{@{}c@{}}
E \vdash \longid \Rightarrow t_1
\qquad
E \vdash \pat \Rightarrow t_2,E'
\\
t_1 \succ t_2 \conarrow t
\qquad
t = c(t_1')\cdots(t_n')
\qquad
\mbox{$c$ extensible}
\end{array}
}{
E \vdash @\longid(\pat) : t \Rightarrow t,E'
}
\end{equation}

\begin{equation}
\frac{
E \vdash \pat \Rightarrow t',E'
\qquad
t = \c{ref}(t')
}{
E \vdash \f{ref}\;\pat : t \Rightarrow t,E'
}
\end{equation}

\begin{equation}
\frac{
E \vdash \pat \Rightarrow t'[t/\alpha]
\qquad
t = \mu \alpha.t'
}{
E \vdash \f{roll}\;\pat : t \Rightarrow t
}
\end{equation}

\begin{equation}
\frac{
E \vdash \pat \Rightarrow t',E'
\qquad
t' \notin V \cup C
\qquad
t = \c{strict}(t')
}{
E \vdash \f{strict}\;\pat : t \Rightarrow t,E'
}
\end{equation}

\begin{equation}
\frac{
E \vdash \pats \Rightarrow t',\cdots,t',E'
\qquad
t = \c{vec}(t')
}{
E \vdash [\pats] : t \Rightarrow t,E'
}
\end{equation}

\begin{equation}
\frac{
E \vdash \pats \Rightarrow t_1,\cdots,t_n,E'
\qquad
t = \{1:t_1,\cdots,n:t_n,\epsilon\}
}{
E \vdash (\pats) : t \Rightarrow t,E'
}
\end{equation}

\begin{equation}
\frac{
E \vdash \flds^\pat \Rightarrow r,E'
\qquad
t = \{\cdots,r\}
}{
E \vdash \{\flds^\pat,\_\} : t \Rightarrow t,E'
}
\end{equation}

\begin{equation}
\frac{
E \vdash \pat_1 \Rightarrow t,E_1
\qquad
E \vdash \pat_2 \Rightarrow t,E_2
\qquad
E' = E_1 \uplus E_2
}{
E \vdash \pat_1\;\f{as}\;\pat_2 : t \Rightarrow t,E'
}
\end{equation}

\begin{equation}
\frac{
E \vdash \pat_1 \Rightarrow t,E'
\qquad
E \vdash \pat_2 \Rightarrow t,E'
}{
E \vdash \pat_1\;|\;\pat_2 : t \Rightarrow t,E'
}
\end{equation}

\begin{equation}
\frac{
E \vdash \pat \Rightarrow t,E'
}{
E \vdash \f{non}\;\pat : t \Rightarrow t,\{\}
}
\end{equation}

\begin{equation}
\frac{
E \vdash \pat \Rightarrow t,E'
\qquad
E \uplus E' \vdash \exp \Rightarrow \c{bool}
}{
E \vdash \pat\;\f{where}\;\exp : t \Rightarrow t,E'
}
\end{equation}

\begin{equation}
\frac{
E \vdash \pat \Rightarrow t,E_1
\qquad
E \uplus E_1 \vdash \decs \Rightarrow E_2
\qquad
E' = E_1 \uplus E_2
}{
E \vdash \pat\;\f{with}\;\decs : t \Rightarrow t,E'
}
\end{equation}



\subsubsection*{Declaration Sequences \hfill
\fbox{$E \vdash \decs \Rightarrow E'$}
}

\begin{equation}
\frac{
E \uplus E_1\uplus\cdots\uplus E_{i-1} \vdash \dec_i \Rightarrow E_i
\qquad
E' = \biguplus E_i
}{
E \vdash \dec_1;\cdots;\dec_n \Rightarrow E'
}
\end{equation}



\subsubsection*{Declarations \hfill
\fbox{$E \vdash \dec \Rightarrow E'$}
}

\begin{equation}
\frac{
E \vdash \pat \Rightarrow t,E'
\qquad
E \vdash \exp \Rightarrow t
}{
E \vdash \pat=\exp \Rightarrow E'
}
\end{equation}

\begin{equation}
\frac{
E \uplus E' \vdash \decs \Rightarrow E'
}{
E \vdash \f{rec}\;\decs \Rightarrow E'
}
\end{equation}



\subsubsection*{Import Sequences \hfill
\fbox{$\vdash \imps \Rightarrow E$}
}

\begin{equation}
\frac{
\vdash \imp_i \Rightarrow E_i
\qquad
E = \biguplus E_i
}{
\vdash \imp_1;\cdots;\imp_n \Rightarrow E
}
\end{equation}



\subsubsection*{Imports \hfill
\fbox{$\vdash \imp \Rightarrow E$}
}

\begin{equation}
\frac{
\vdash \id \Rightarrow t,E
}{
\vdash \id\;\f{from}\;s \Rightarrow E
}
\end{equation}



\subsubsection*{Components \hfill
\fbox{$\vdash \com$}
}

\begin{equation}
\frac{
\vdash \imps \Rightarrow E_1
\qquad
E_1 \vdash \decs \Rightarrow E_2
\qquad
E_1 \uplus E_2 \vdash \flds^\id \Rightarrow r
}{
\vdash \f{import}\;\imps\;\f{let}\;\decs\;\f{in}\;\flds^\id
}
\end{equation}



%%%%%%%%%%%%%%%%%%%%%%%%%%%%%%%%%%%%%%%%%%%%%%%%%%%%%%%%%%%%%%%%%%%%%%%%%%%%%%%%
\end{document}
%%%%%%%%%%%%%%%%%%%%%%%%%%%%%%%%%%%%%%%%%%%%%%%%%%%%%%%%%%%%%%%%%%%%%%%%%%%%%%%%
