\documentclass[a4paper,titlepage]{article}
\usepackage{german}
\setlength{\textwidth}{14cm}
\setlength{\oddsidemargin}{1cm}

\usepackage{graphics}
\newcommand{\showimage}[1]{\begin{center}\includegraphics{#1.eps}\end{center}}
%\newcommand{\showimage}[1]{\begin{picture}(100,100)\end{picture}}

\usepackage{verbatim}
\newenvironment{code}{\verbatim}{\endverbatim}
\usepackage{shortvrb}
\MakeShortVerb{\#}

\begin{document}

\begin{titlepage}
  \begin{small}\noindent Fachrichtung 6.2 -- Informatik\\
  Naturwissenschaftlich-Technische Fakult"at I\\
  Universit"at des Saarlandes\\
  \end{small}
  \vspace{2cm}

  \begin{center}

    {\LARGE Eine GTK-Schnittstelle f"ur Alice}
    \vspace{1cm}

    Robert Grabowski\\
    #grabow@ps.uni-sb.de#\\
    Programming Systems Lab

    \vspace{2cm}
    {\Large Fortgeschrittenen-Praktikum}

    \vspace{1cm}
    Angefertigt unter der Leitung von\\
    Prof. Dr. Gert Smolka

    \vspace{1cm}
    Betreut von\\
    Thorsten Brunklaus
    
    \vspace{4cm}
    \today

  \end{center}

\end{titlepage}

\tableofcontents

\section{Einleitung}

\subsection{Inhalt des Berichts}

In diesem Bericht werden Entwurf und Implementierung der 
GTK+-Schnittstelle f"ur Alice beschrieben. Mit GTK+ l"asst sich relativ einfach
eine grafische Benutzeroberfl"achen (GUI) in C erstellen. 
Durch die Schnittstelle soll
dies auch dem Alice-Programmierer m"oglich sein.

Im Folgenden werden die Schnittstelle und GTK+ kurz vorgestellt.
Daraufhin wird der Entwurf der Schnittstelle er"ortert, wobei auch auf
allgemeine, nicht GTK-spezifische Konzepte hingewiesen wird.
Es folgt eine Beschreibung der Implementierung, bevor der Bericht mit einem
Ausblick auf weiterf"uhrende Arbeiten schlie"st.

Dieser Bericht geht nicht auf die eigentliche GUI-Programmierung mit GTK+ ein.
Informationen dazu findet man auf \cite{wwwgtkorg}.

\subsection{Die Schnittstelle}

Als Grundlage f"ur diese Arbeit dienen zwei bereits existierende
Schnittstellen f"ur GTK+ 1.2 von Thorsten Brunklaus:
eine Anbindung an Mozart, und eine an Alice auf der Mozart-VM.
Die hier besprochene Schnittstelle bindet GTK+ 2.x an Alice auf der SEAM-VM.
Sie ist in der Benutzung weitestgehend kompatibel mit der bereits existierenden
Alice-Anbindung. Die gr"osten Unterschiede liegen bei GTK+ selbst,
bedingt durch den Versionssprung.

Alle erw"ahnten Schnittstellen sind sowohl unter Windows als auch unter
Linux lauff"ahig. Benutzt werden die GTK+-Schnittstellen unter anderem von
den Inspector-Tools \cite{inspector}.

\subsection{GTK+}

GTK+ ist eine Bibliothekssammlung zum komfortablen Erstellen von grafischen
Benutzeroberfl"achen. Urspr"unglich im Jahr 1996 nur zur Verwendung im
Grafikprogramm GIMP geschrieben\footnote{GTK steht f"ur ``GIMP Toolkit''}, 
bildet es
inzwischen die Grundlage sehr vieler Anwendungen, insbesondere des 
GNOME-Desktops. GTK+ ist komplett
in C geschrieben, doch bei der Entwicklung wurde von Anfang an darauf geachtet,
m"oglichst gute Voraussetzungen f"ur die Anbindung an andere
Programmiersprachen zu schaffen.

\subsubsection*{Konzepte} 
GTK+ basiert auf folgenden grundlegenden Konzepten:

\begin{itemize}
\item \emph{Widgets.}
      Ein Widget ist ein Objekt, das ein
      Element der Benutzeroberfl"ache darstellt, wie
      Fenster, Schaltfl"achen und Men"us.
      Widgets sind in einem Pseudo-Klassensystem hierarchisch angeordnet
      und besitzen Attribute, die
      z.B. Aussehen und Verhaltensweisen beeinflussen.

\item \emph{Ereignisbehandlungsroutinen.}
      Diese auch ``Event Handler'' oder
      ``Callbacks'' genannten Funktionen reagieren auf bestimmte Ereignisse, 
      wie z.B. Mausbewegungen oder Tastatureingaben.

\item \emph{Ereignisgesteuerte Hauptschleife.} 
      Diese "uberpr"uft periodisch, ob Ereignisse
      vorliegen,
      und ruft ggf. die f"ur ein Ereignis registrierten Callback-Funktionen
      auf.

\item \emph{Speicherverwaltung.} Nicht mehr ben"otigte Objekte werden von
      GTK+ automatisch entfernt.

\end{itemize}

\subsubsection*{Aufbau}
GTK+ 2.0 besteht aus den folgenden Bibliotheken:

\begin{itemize}
\item GLib -- grundlegende Funktionen wie Klassensystem und
      Speichermanagement
\item GDK -- plattformunabh"angige Abstraktion der Grafikfunktionen
      des Betriebssystems
\item Pango -- abstrahiert Schriftartenfunktionen des Betriebssystems
\item ATK -- stellt Funktionen f"ur Eingabehilfen zur Verf"ugung
\item GTK -- enth"alt die eigentliche Sammlung von Widgets
\end{itemize}

In diesem Alice-GTK+-Binding werden nur Schnittstellen f"ur die Bibliotheken
GTK und GDK generiert, denn auf die anderen Bibliotheken muss in der Regel beim
Programmieren von GTK+-Anwendungen nicht direkt zugegriffen werden.\\[\parskip]

Dar"uberhinaus enth"alt das Binding eine Schnittstelle f"ur das
GnomeCanvas-Widget, welches eine abstrakte
Fl"ache zum Zeichnen von Texten, Bilder und primitiven
Objekten darstellt. Die GnomeCanvas-Bibliothek ist eigentlich nicht Teil
von GTK+, sondern geh"ort bereits zum Gnome-Projekt. Eine Schnittstelle ist 
dennoch erforderlich, da diverse Alice-Tools
wie der Inspector dieses Widget zentral zum Anzeigen von Informationen
nutzen.

\subsubsection*{GTK und GTK+}

Obwohl die Sammlung der Bibliotheken ``GTK+'' hei"st und ``GTK'' nur eine
einzelne Bibliothek ist, wird im Folgenden aus Gr"unden der Vereinfachung
und Lesbarkeit immer nur von ``GTK'' gesprochen, womit das gesamte
Paket gemeint ist.


\section{Entwurf}

\subsection{Alice-Schnittstellen f"ur C-Bibliotheken}

In diesem Abschnitt werden allgemeine Konzepte f"ur die Entwicklung von
Alice-Bindings er"ortert. Gleichzeitig wird erkl"art,
wie gewisse Probleme speziell in der GTK-Schnittstelle gel"ost sind.

Eine C-Bibliothek besteht im Wesentlichen eine Reihe von Funktionen.
Zus"atzlich werden komplexe Datentypen deklariert, die bei der Verwendung
mit den Funktionen sinnvoll sind. Ein Binding hat die Aufgabe, diese
Funktionen und Datentypen dem Alice-Programmierer zur Verf"ugung zu stellen.

\subsubsection*{Funktions-Abbildung}

Weil Bibliotheksfunktionen nicht direkt von Alice aufrufbar sind,
wird f"ur alle Funktionen ein Wrapper geschrieben, der in Alice
sichtbar ist und der den eigentlichen Funktionsaufruf t"atigt.

Ein Wrapper muss zudem ein Datenmarshalling durchf"uhren, d.h. vor
dem eigentlichen Funktionsaufruf Alice-Werte in C-Werte umwandeln und
umgekehrt. Die genaue Arbeitsweise h"angt nat"urlich davon ab, wie mit
C-Datentypen in Alice umgegangen wird.

\subsubsection*{Datentyp-Abbildung}

Es gibt zwei ``extreme'' M"oglichkeiten f"ur die Datentyp-Abbildung.

Eine M"oglichkeit besteht darin, das gesamte C-Typsystem durch eine Reihe von
abstrakten Alice-Typen darzustellen. In diesem Fall muss die Schnittstelle
aber Operatoren f"ur Werte dieser abstrakten Typen zur Verf"ugung stellen,
wie etwa Zuweisungs- und Vergleichsoperatoren f"ur C-Integers.
Dieser Ansatz wird beispielsweise durch Matthias Blumes
``Foreign Function Interface'' f"ur SML implementiert \cite{blume}.
Zwar k"onnen auf diese Weise alle Aspekte des C-Typsystems sehr genau
abgebildet werden, jedoch wird durch den indirekten Zugriff auf C-Werte die
Programmierung unter Alice deutlich erschwert.

Als Alternative k"onnte man versuchen, f"ur alle C-Typen ein m"oglichst
"ahnliches Alice-Pendant zu finden. Das erleichert zwar den Umgang mit
der Schnittstelle, allerdings entsteht
dadurch im Binding ein ineffizienter Overhead f"ur die st"andige
Datenkonversion. Vor allem aber gibt es nicht immer
ein passendes Typ-"Aquivalent in Alice.

Das GTK-Binding geht einen Mittelweg zwischen beiden genannten M"oglichkeiten:
primitive Datentypen und Enumerationstypen werden auf "ahnliche Alice-Typen
abgebildet, w"ahrend Verbundtypen in Alice abstrakt bleiben.
Gr"unde f"ur diese Entscheidung und Implementationsdetails werden in Abschnitt
\ref{Implementierung} erkl"art.

\subsubsection*{Die automatische Generierung und ihre Grenzen}

Sobald man sich auf eine bestimmte Abbildung festgelegt hat, liegt es nahe,
das Binding anhand der Headerdateien der C-Bibliothek automatisch generieren zu
lassen. Leider enthalten reine C-Deklarationen manchmal zuwenig semantische
Informationen, um eine m"oglichst akkurate Alice-Schnittstelle generieren
zu k"onnen.\footnote{Ein Beispiel daf"ur sind Ein- und Ausgabeparameter, die
weiter unten besprochen werden.} Weil auch dies auch bei GTK der Fall
ist und eine Annotierung mit Meta-Daten nicht vorliegt, macht der Generator
an einigen Stellen bestimmte Annahmen, die zumindest f"ur die GTK-Bibliotheken
zutreffen. Ausserdem kann f"ur Einzelf"alle eine gesonderte Code-Generierung
festgelegt werden.

Die so generierte Alice-Schnittstelle ist im Wesentlichen
eine 1:1-Abbildung der C-Biblio\-thek. Alice-Programme, die die Schnittstelle
benutzen, sind also reichlich imperativ und auf Seiteneffekten beruhend.
Der Ausblick am Ende dieses Berichts erl"autert Ideen f"ur eine h"ohere,
funktionale Schicht.

Die direkte Abbildung auf unterer Ebene bringt noch ein weiteres Problem
mit sich: Wenn die Bibliothek in Alice genauso benutzt wird wie in C, 
k"onnen unter Umst"anden fundamentale Alice-Konzepte wie die Nebenl"aufigkeit
ausgehebelt werden. Daher m"ussen einige Teile der
Schnittstelle auf jeden Fall von Hand geschrieben werden.
Bei GTK sind dies die Hauptschleife, die Ereignisbehandlung sowie die
Garbage Collection, welche im n"achsten Abschnitt erkl"art werden.

\subsection{Hauptschleife und Ereignisbehandlung}

\subsubsection*{GTK-Programmierung in C}

Abbildung \ref{EventsC} zeigt den Kontrollfluss eines in C
geschriebenen GTK-Programms. Dabei laufen die folgenden Schritte ab:
\begin{figure}
\showimage{events-c}
\caption{Ereignisbehandlung in C}
\label{EventsC}
\end{figure}

\begin{enumerate}
\item Es werden sogenannte ``Callback-Funktionen'' bei GTK registriert, die
      bei bestimmten Ereignissen reagieren sollen.
\item Das Hauptprogramm "ubergibt die Kontrolle an die GTK-Hauptschleife,
      welche erst zum Ende des Programms zur"uckkehrt.
\item Die Hauptscheife "uberpr"uft regelm"a"sig, ob Ereignisse vorliegen.
\item Wenn ein Ereignis eintritt (z.B. der Benutzer eine Schaltfl"ache 
      dr"uckt), werden die f"ur dieses Ereignis registrierten
      ``Callback-Funktionen'' aufgerufen.
\end{enumerate}

Zu beachten ist dabei, dass die Ereignisse synchron abgearbeitet werden,
d.h. erst wenn die Callback-Funktion zur"uckgekehrt ist, kann die Hauptschleife
wieder neue Ereignisse abarbeiten.

Der naheliegende Ansatz, dieses Konzept unver"andert auf Alice-Programme
zu "ubertragen, muss aus zwei Gr"unden scheitern.
Zum einen k"onnen Alice-Callback-Funktionen nicht bei GTK registriert werden,
geschweige denn von C aus aufgerufen werden.
Zum anderen sind Aufrufe von C-Funktionen f"ur die Alice-VM atomar. 
W"ahrend die
Programmkontrolle bei der Hauptschleife liegt, was fast immer der Fall ist,
ist die virtuelle Maschine blockiert. Dadurch kann sie keine weiteren
Alice-Threads mehr abarbeiten, und die Nebenl"aufigkeit ist nicht mehr
sichergestellt.


\subsubsection*{Ereignisbehandlung in Alice}

Abbildung \ref{EventsAlice} zeigt, wie die Probleme in Alice gel"ost werden.
\begin{figure}
\showimage{events-alice}
\caption{Ereignisbehandlung in Alice}
\label{EventsAlice}
\end{figure}

\begin{itemize}
\item Teile der Hauptschleife werden nach Alice ``hochgezogen''.
\item Alice-Callbacks-Funktionen werden indirekt "uber ein Ereignisstrom-Modell
      aufgerufen.
\end{itemize}

Die Hauptschleife wird in einen C- und einen Alice-Teil aufgeteilt.\footnote{
Diese Aufteilung ist kein ``Hack'', sondern explizit von den
GTK-Entwicklern als M"oglichkeit vorgesehen.}
In einem eigenen Alice-Thread wird dabei das Verhalten
der GTK-Hauptschleife ``simuliert'': Der Thread ruft periodisch
GTK-Funktionen auf, die "uberpr"ufen, ob Ereignisse vorliegen bzw. diese
abarbeiten. Durch diese Konstruktion kehrt der Programmfluss immer wieder
zu Alice zur"uck, wodurch die VM nicht blockiert wird.

Die Realisierung des indirekten Aufrufs der Alice-Callback-Funktion 
ist etwas komplizierter:

\begin{enumerate}
\item Soll eine Funktion f"ur ein Ereignis registriert werden,
      wird f"ur sie eine eindeutige ID generiert. 
      Bei GTK wird statt der Funktion ein in C geschriebener
      ``generischer Marshaller'' registriert, und zwar so, dass er bei jedem
      Aufruf die ID als Argument erh"alt.
\item Wird das Ereignis ausgel"ost, ruft GTK den Marshaller mit der ID auf.
      Dieser schreibt die ID auf den Ereignisstrom.
\item Auf Alice-Seite wartet ein ``listener-Thread'' auf Daten vom Strom.
      Dieser Thread kann anhand einer ankommenden ID die zugeh"orige
      Alice-Callback-Funktion herausfinden und aufrufen.
\end{enumerate}

Es gibt einen wichtigen Unterschied
zur Ereignisbehandlung bei einem C-Programm: Die Ereignisse werden asynchron
abgearbeitet, d.h. die Hauptschleife kann bereits ein Ereignis bearbeiten,
w"ahrend ein anderes noch in einer Alice-Callback-Funktion abgearbeitet wird.
Durch den Strom werden die Ereignisse werden aber trotzdem nacheinander
in der richtigen Reihenfolge abgearbeitet.

\subsection{Garbage Collection}

GTK+ besitzt eine eigene interne Speicherverwaltung, die daf"ur sorgt,
das GTK-Objekte entfernt werden, sobald sie nicht mehr ben"otigt werden.
Dies wird "uber einen Z"ahler realisiert, der festh"alt, wieviele
Referenzen auf das Objekt bestehen. Sobald der Z"ahler auf 0 sinkt,
entfernt GTK das Objekt aus dem Speicher.

Das Problem dabei ist, das der Z"ahler zun"achst nur bibliotheksintern
verwaltet wird. Nachdem eine GTK-Funktion eine Referenz
auf ein Objekt geliefert hat und diese Referenz in Alice sichtbar ist,
kann es passieren, dass GTK das Objekt intern nicht mehr ben"otigt und
daher entfernt wird, weil die Bibliothek keine Kenntnis von der Alice-Referenz
besitzt. Wird die Alice-Referenz dann weiterhin benutzt, kann das fatale
Folgen haben.

Daher erh"oht die Schnittstelle selbst den Referenzz"ahler automatisch:

\begin{itemize}
\item Sobald eine GTK-Funktion eine Objekt-Referenz liefert und es noch keine
      Referenz von Alice auf das Objekt gibt, wird der GTK-Referenzz"ahler
      um eins erh"oht.
\item Sobald unter Alice letzte Referenz auf das Objekt verschwindet, wird
      dessen Z"ahler wieder um eins verringert (Fussnote: ueber AliceFinalis)
\end{itemize}

Solange eine Referenz auf ein Objekt in Alice also sichtbar ist, muss der
GTK-Referenz\-z"ahler mindestens 1 sein, was GTK am Entfernen des Objekts
hindert.


\section{Implementierung}
\label{Implementierung}

\subsection{Allgemeiner Aufbau der Schnittstelle}

\begin{figure}
\showimage{layout}
\caption{Aufbau der Schnittstelle}
\label{Layout}
\end{figure}
Abbildung \ref{Layout} zeigt den allgemeinen Aufbau der GTK-Schnittstelle.
Sowohl die generierte Schnittstelle als auch der handgeschriebenen ``Kern''
besitzen einen in C implementierten Teil, der die eigentliche Kommunikation
mit der Bibliothek abwickelt, als auch eine in Alice implementierte
Schicht, die von der unteren Ebene abstrahiert und u.a. eine h"ohere
Typsicherheit bietet. F"ur den Zugriff auf die C-Ebene von der Alice-Seite
und umgekehrt sorgt das Foreign Function Interface (FFI) von Alice.

\subsection{Generator}

\begin{figure}
\showimage{generator}
\caption{Generierungs-Phasen}
\label{Generator}
\end{figure}
Die Generierung l"auft in mehreren Phasen ab (siehe dazu auch
Abbildung \ref{Generator}):
\begin{itemize}
\item Mithilfe des C-Pr"aprozessors und weiterer, auf Textebene arbeitender
      Programme werden die Deklarationen in allen
      Header-Dateien der GTK-Bibliotheken in eine Datei zusammengefasst.
\item Die C-Deklarationen werden mithilfe der
      CKIT-Bibliothek\footnote{Das CKIT \cite{ckit} kann beliebige C90-konforme
      C-Programme parsen. Es ist Bestandteil der SML/NJ-Suite.}
      geparst und in einem abstrakten Syntaxbaum dargestellt.
\item Die CKIT-Darstellung f"ur unsere Zwecke aufgrund der reichhaltigen
      Annotationen sehr unhandlich und ineffizient. Daher wird sie in eine
      eigene Zwischenrepr"asentation umgewandelt, welche genau diejenigen
      Informationen enth"alt, die f"ur die Generierung notwendig sind.
\item Anhand dieser Repr"asentation kann schlie"slich der eigentliche
      Schnittstelle (Alice- und C-Code) generiert werden.
\end{itemize}

\subsection{Abbildung der Datentypen}

F"ur welche konkreten Elemente kann das Binding "uberhaupt eine Schnittstelle
liefern? Neben den primitiven Datentypen von C sind dies die in der Bibliothek
deklarierten Funktionen, Enumerationstypen, Verbundtypen, Typ-Aliasnamen
und externen Variablen.

\paragraph{Primitive Datentypen}

In diesem Binding werden primitive C-Datentypen in der Regel auf m"oglichst
"ahnliche primitive Alice-Typen abgebildet. Einen "Uberblick gibt
Tabelle \ref{PrimTab}.

%\renewcommand{\topfigrule}{\vspace{0.5cm}\noindent\rule{\columnwidth}{0.4pt}}
\begin{table}
\begin{center}
  \begin{tabular}{|c|c||c|c|}
  \hline \textbf{C-Typ}     & \textbf{Alice-Typ} &
         \textbf{C-Typ}     & \textbf{Alice-Typ}\\
  \hline #void#             & #unit# & #char*#            & #string#     \\
  \hline (ganzzahliger Typ) & #int#  & \emph{t}#[]#       & \emph{t}# vector#\\
  \hline (Flie"skomma-Typ)  & #real# & \emph{t}#*#        & #object# \\
  \hline #gboolean#         & #bool# & #GList*#/#GSList*# & #object list# \\
  \hline
  \end{tabular}
\end{center}
\caption{Abbildung primitiver Datentypen}
\label{PrimTab}
\end{table}

\begin{itemize}
\item
#object# ist dabei ein abstrakter Datentyp, der einen C-Zeiger repr"asentiert.
Es ist ohne die GTK-Bibliothek nicht m"oglich, ein Wert vom Typ #object# zu
konstruieren oder zu dereferenzieren, lediglich ein Vergleich ist m"oglich.
Mehr ist auch nicht n"otig, denn f"ur Zeiger auf Verbundtypen gibt es
Zugriffsfunktionen, und Zeiger auf andere Werte werden als Ausgabeparameter
betrachtet (siehe unten).

\item 
#GList# und #GSList# sind GTK-Datenstrukturen f"ur die Verwaltung von Listen.

\item
Bei Strings, Arrays und Listen kann sich in C der Zustand/Inhalt "andern,
ohne dass diese "Anderungen auf Alice-Seite reflektiert werden. Anhand der
Konzeption von GTK kann jedoch davon ausgegangen werden, dass diese Typen
nur als ``Container'' f"ur den Datentransport benutzt werden, und etwaige
"Anderungen durch eine GTK-Funktion in einem neuen ``Container'' zur"uckkommen.
\end{itemize}

\paragraph{Enumerationstypen}

Diese werden auf Alice-Datatypes mit nullstelligen
Konstruktoren abgebildet. Dadurch wird eine neue Typsicherheit eingef"uhrt,
die unter C nicht vorhanden war, da dort alle Enumerationskonstanten im Grunde
Integer-Werte sind.\footnote{Die Konstruktor-Integer-Konversion kostet
nat"urlich Zeit. Bei einer direkte Implementierung als Alice-Integerkonstanten
(Variablen) w"urde aber die Typsicherheit fehlen; zudem sind z.B.
pattern matchings "uber Variablen nicht m"oglich.} Beispiel:\\[\parskip]

\begin{minipage}[t]{5cm}
\begin{code}
typedef enum {
  GTK_WINDOW_TOPLEVEL,
  GTK_WINDOW_POPUP
} GtkWindowType;
\end{code}
\end{minipage}
\begin{minipage}[t]{5cm}
\begin{code}
datatype GtkWindowType =
  WINDOW_TOPLEVEL
| WINDOW_POPUP
\end{code}
\end{minipage}

\paragraph{Verbundtypen}

Bei Verbundtypen\footnote{Das sind in C ``structs'' und ``unions''.} 
liegt es nahe,
sie als Alice-Records zu repr"asentieren. Das ist aber sehr teuer, zumal in C
nur Zeiger auf Verbunde, nicht jedoch Verbunde selbst
an Funktionen "ubergeben werden k"onnen.
Daher reicht der Typ #object# f"ur die Benutzung von Verbunden aus.
Zudem kann und sollte der Programmierer GTK-Verbunde nie selbst
erstellen oder entfernen, sondern nur die entsprechenden Funktionen benutzen.
Die Schnittstelle muss sich also nicht um eigene Konstruktoren und Destruktoren
k"ummern. Allerdings erfordert die GTK-Programmierung bisweilen einen direkten
Zugriff auf Felder eines Verbunds zugreift. Daher werden f"ur jeden
Verbundtyp Feldzugriffsfunktionen generiert.

\paragraph{Typaliasnamen und externe Variablen}

Diese werden von
der Schnittstelle nicht behandelt. Eine Abbildung von Typaliasnamen auf
type-Deklarationen in Alice w"urde erst richtig Sinn machen, wenn man
statt des nicht n"aher spezifizierten #object# einen polymorphen
``$\alpha$ #pointer#'' benutzen w"urde. Ein Zugriff auf externe Variablen
ist bei der GTK-Programmierung in aller Regel nicht n"otig.

\subsection{Funktionswrapper}

F"ur jede Funktion wird ein Funktionswrapper generiert, der das
Datenmarshalling durch\-f"uhrt, d.h. Werte eines Alice-Typs in Werte eines
C-Typs konvertiert und umgekehrt. Der Alice-Funktionstyp folgt dabei der
C-Funktionsdeklaration; mehrere Argumente werden als Tupel "ubergeben.
Die Funktionsnamen sind nach der ``lower CAML case''-Konvention
umgewandelt. Beispiel:

\begin{code}
void gtk_scale_set_draw_value (GtkScale*, gboolean)
val scaleSetDrawValue : object * bool -> unit
\end{code}

\subsubsection*{Ein- und Ausgabeparameter}

Manche GTK-Funktionen nehmen Zeiger auf primitive Typen entgegen:

\begin{code}
void gtk_entry_get_layout_offsets (GtkEntry*, int*, int*)
\end{code}

Mittels solcher Zeiger kann die Funktion zus"atzliche Ausgabewerte
zur"uckliefern. Da in Alice mehrere R"uckgabewerte (als Tupel) m"oglich sind,
betrachtet die Schnittstelle alle Zeiger mit Ausnahme von Verbundzeigern,
Strings und Listen als Ausgabeparameter und erstellt daher den folgende
Wrapper:

\begin{code}
val entryGetLayoutOffsets : object -> int * int
\end{code}

M"oglicherweise greift die GTK-Funktion aber zus"atzlich noch lesend auf 
einen solchen Wert zu. Aufgrund fehlender semantischer Informationen kann
der Generator nicht wissen, ob es sich um einen kombinierten Ein- und
Ausgabe-Parameter handelt.
Daher wird f"ur eine solche Funktion ein weiterer Wrapper generiert,
der jeden Ausgabeparametern zus"atzlich als Eingabeparameter betrachtet:

\begin{code}
val entryGetLayoutOffsets' : object * int * int -> int * int
\end{code}

Es liegt dann am Programmierer, die richtige Variante verantwortungsbewusst
zu benutzen.

\subsubsection*{Funktionen mit variablen Argumentlisten}

Im Bereich der Sprachanbindungen ist dies eines der d"ustersten Kapitel,
welches hier aber trotzdem nicht verschwiegen werden soll, denn leider
macht GTK intensiv von diesem ``Feature'' Gebrauch.

Funktionen mit variablen Argumentlisten k"onnen beliebig viele Argumente
erhalten, und jedes Argument kann von einem beliebigen Typ sein.
Die "Ubergabe der Argumente von Alice an die Schnittstelle l"asst sich dabei
noch halbwegs elegant gestalten. Auf Alice-Seite wird eine Liste vom Typ
#arg list# "ubergeben, wobei #arg# die variablen Typen repr"asentiert:

\begin{verbatim}
datatype arg = INT of int | BOOL of bool | STRING of string | ...
\end{verbatim}

Das eigentliche Problem besteht darin, dass zur "Ubersetzungszeit
\emph{der Schnittstelle} Anzahl und Typen der Argumente bekannt sein m"ussen.
Sie werden aber fr"uhestens zur "Ubersetzungszeit des
Alice-Programms bekannt.

Es gibt in C keine plattformunabh"angige M"oglichkeit,
variable Argumentlisten zur Laufzeit zu bauen und einer Funktion zu "ubergeben.
Die GTK-Schnittstelle l"ost das Problem momentan auf eine Art, die
auf x86-Plattformen unter Windows und Linux keine Probleme
zu bereiten scheint. Man kann die Plattformunabh"angigkeit noch ``steigern'',
indem auf Alice-Seite solche Funktionen mit h"ochstens einem Argument
aufgerufen werden.

Eine bessere L"osung w"are es, eine spezielle low-level-Bibliothek
wie ``libffi'' zu benutzen, die das plattformabh"angige dynamische "Ubergeben
von variablen Argumenten abstrahiert.


\section{Ausblick}

Weiterf"uhrende Arbeiten k"onnten sowohl ``in die Tiefe'' gehen und die
GTK-Schnittstelle ausbauen, als auch ``in die Breite'' und die automatische
Generierung verallgemeinern.

\subsection*{Funktionale GUI-Programmierung}

Die hier vorgestellte Schnitstelle bietet eine Anbindung auf niedriger Ebene.
Durch die 1:1-Abbildung sieht die GTK-Programmierung unter Alice "ahnlich
aus wie unter C. Diese unter Alice unelegante Programmierweise kann
nur mit einer funktionalen, abstrahierenden Schicht "uber der
eigentlichen GTK-Schnittstelle umgangen werden.

Einen Anfang daf"ur bildet die GtkBuilder-Struktur von Thorsten Brunklaus.
Das Layout ganzer Fensterinhalte kann als Wert mittels Konstruktoren
repr"asentiert werden. Daraufhin gen"ugt der Aufruf einer einzigen Funktion,
welche diesen Wert anschlie"send in eine Reihe von GTK-Befehlen "ubersetzt.

W"unschenswert w"are es, spezielle Features der Programmiersprache Alice
gleich zu nutzen. So k"onnten z.B. "ahnliche Mitteilungsfenster
in einem Funktor zusammengefasst werden, und dieser k"onnte wiederum als
Package im Netz zur allgemeinen Verf"ugung stehen. Weitere Ideen
f"ur eine funktionale GUI-Programmierung bietet das Fudgets-Konzept
\cite{fudgets}. 

In jedem Fall ist eine solche funktionale Schnittstelle
nicht mehr automatisch generierbar.

\subsection*{Allgemeines Genererierungstool}

Auch wenn der Generator in der jetzigen Form bereits m"oglichst unabh"angig
von GTK arbeitet, macht er dennoch einige GTK-spezifische Annahmen, die
eine Anwendung f"ur andere Bibliotheken erschweren.

Daher w"are es n"utzlich, diese Annahmen zu parametrisieren, um einen
allgemeinen Generator zu erhalten, der f"ur beliebige C-Bibliotheken
eine Anbindung an Alice generieren kann.

\begin{thebibliography}{99}
  \bibitem{wwwgtkorg}GTK+ Homepage\\
                     #http://www.gtk.org/#

  \bibitem{inspector}Bernadette Blum, Marvin Schiller.
                     \emph{Ein Browser f"ur Alice} (Fopra), 2003.\\
                     #http://www.ps.uni-sb.de/~schiller/inspector.html#

  \bibitem{blume}Matthias Blume. \emph{No-Longer-Foreign: Teaching an
                 ML compiler to speak C ``natively''}, 2001.\\
                 #http://people.cs.uchicago.edu/~blume/papers/nlffi.pdf#

  \bibitem{ckit}CKIT f"ur SML/NJ\\
                #http://www.smlnj.org/doc/ckit/#
                 
  \bibitem{fudgets}Magnus Carlsson, Thomas Hallgren.
                   \emph{Fudgets -- Purely Functional Processes with
                   applications to Graphical User Interfaces} (PhD thesis), 
                   1998.\\
                   #http://www.cs.chalmers.se/~hallgren/Thesis#
\end{thebibliography}

\end{document}
